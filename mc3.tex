% See author's guidelines for Methods in Ecology and Evolution:
% http://www.methodsinecologyandevolution.org/view/0/authorGuidelines.html

\documentclass[12pt]{article}

% One-inch margins
\usepackage{fullpage}

% No orphans or widows
\usepackage[all]{nowidow}

% Additional math features
\usepackage{amsmath}

% Use mee bibliography style
\usepackage{natbib}
\bibliographystyle{mee}

% Definitions to make it easier to type
\newcommand{\MCMCMC}{(MC)$^{3}$}

\begin{document}

% Double-spaced
\baselineskip 24pt

\begin{flushleft}

{\Large\textbf{Metropolis coupled Markov chain Monte Carlo
    for detecting rate shifts on phylogenetic trees}}

% Short title must be <= 45 characters (current: 47, including spaces)
\textbf{Short title:} Metropolis coupled MCMC for rate shift detection

% Recommended: 6000-7000 words, including tables, figure captions, references
\textbf{Word count:} 0

Carlos J. R. Anderson$^{1}$ and
Daniel L. Rabosky$^{1,*}$

$^{1}$Department of Ecology and Evolutionary Biology,
    University of Michigan, Ann Arbor, MI 48109, USA

$^{*}$Corresponding author: Daniel L. Rabosky (drabosky@umich.edu)

\end{flushleft}


\pagebreak[4]


% Maximum of 350 words
\section*{Abstract}

% Suppress indentation for the following paragraphs
{\setlength{\parindent}{0cm}

\textbf{1.}
BAMM (Bayesian Analysis of Macroevolutionary Mixtures) is a computer program
that uses Markov chain Monte Carlo (MCMC) to model complex dynamics
of speciation, extinction, and trait evolution on phylogenetic trees.
%
Previous versions of BAMM used a single Markov chain
to explore the landscape of models and their parameter values.
%
A single chain, however, may get stuck in local optima,
resulting in less mixing and more time needed for convergence.

\textbf{2.}
Here we describe an extension to BAMM
that implements Metropolis coupled MCMC [\MCMCMC].
%
In \MCMCMC, additional chains are introduced
that are able to explore more of the model landscape.
%
Swaps between chains may occur periodically,
so if a chain is stuck in a local optimum,
it may immediately jump to another area of the landscape.
%
In addition, we describe multi-core \MCMCMC,
which is able to run each chain on a separate CPU.

\textbf{3.}
We find that in both simulated data sets and an empirical data set,
\MCMCMC\ mixes better and reduces the time to convergence
compared to a single-chain MCMC.
%
Greater than four chains results in diminishing returns.

\textbf{4.}
\MCMCMC\ greatly improves the reliability of results in BAMM,
and we recommend its use in all future studies of macroevolutionary dynamics.
}

\begin{flushleft}
\textbf{Key-words:} BAMM, Bayesian, macroevolution, phylogeny, skink, speciation
\end{flushleft}


\pagebreak[4]


% State the reason for the work, the context and the hypotheses being tested
\section*{Introduction}

The introduction will go here, including discussion of BAMM.


% Include sufficient details for the work to be repeated
\section*{Materials and methods}

\subsection*{Metropolis coupled MCMC}

We followed \citet{alt04} for the implementation of \MCMCMC\ in BAMM.
%
For each Markov chain $i \in \{1, 2, \dots\}$, its temperature was set to
$\beta_i = [1 + \Delta T \times (i - 1)]^{-1}$,
where $\Delta T$ is the temperature increment parameter.
%
For example, if there are 4 chains and $\Delta T$ is 0.1,
the temperature of each chain is 1, 0.9091, 0.8333, and 0.7692.
%
The cold chain always has a temperature of 1.
%
The value of $\Delta T$ should be greater than 0
and chosen such that the probability of accepting a swap
is between 20\% and 60\% \citep{alt04}.


The temperature of chain $i$ goes into the calculation
of the acceptance probability $\alpha_i$ for a move proposal.
%
For moves that do not involve changes in the dimensionality of the model,
\[\alpha_i = \text{min}\left\{ 1,
    \left(
    \frac{f(\theta_i')}{f(\theta_i)} \times
    \frac{\pi(\theta_i')}{\pi(\theta_i)}
    \right)^{\beta_i} \times
    \frac{q(\theta_i | \theta_i')}{q(\theta_i' | \theta_i)}
\right\}\]
where $\theta_i$ and $\theta_i'$ are parameter vectors
corresponding to the current and proposed states for chain $i$,
$f$ and $\pi$ are the corresponding likelihood and prior density functions,
and $q(\theta_i' | \theta_i)$ is the relative probability
of proposing a move to parameter vector $\theta_i'$,
given that the current state is $\theta_i$.
%
A similar calculation was done for the acceptance probability for proposals
that change the dimensionality of the model.

After a certain number of generations, two randomly chosen chains $j$ and $k$
are swapped with acceptance probability
\[\alpha = \text{min}\left\{ 1,
    \left(\frac{f(\theta_k)}{f(\theta_j)}\right)^{\beta_j} \times
    \left(\frac{f(\theta_j)}{f(\theta_k)}\right)^{\beta_k}
\right\}\]


\subsection*{Performance analysis}

We tested the performance of our implementation of \MCMCMC\ 
with both simulated trees and an empirical tree of Australian skinks.
%
Performance was assessed by comparing the effective size values
between runs without \MCMCMC\ and those with \MCMCMC.


For the simulated trees, we used 100 random trees generated by \citet{rab14plos}
under a pure-birth process at the root and four shifts
to diversity-dependent speciation-extinction processes.
%
We used BAMM (version 2.0.1) to model rates of species diversification
across each tree with and without \MCMCMC.
%
Runs with \MCMCMC\ were configured with four chains,
the temperature increment parameter $\Delta T$ was set to 0.1,
and the swap period was set to 1,000 generations.
%
Runs went for 25 million generations,
sampling every 25,000 to produce a total of 1,000 samples.


As an emprical tree, we used the maximum clade credibility (MCC) tree
of Australian sphenomorphine skinks reconstructed by \citet{rab14sysbio}.
%
We used BAMM (version 2.0.0) to model rates of species diversification
across the tree with and without \MCMCMC.
%
We performed 25 runs with a single Markov chain [i.e., without \MCMCMC]
and 25 runs with four chains [i.e., with \MCMCMC].
%
The value of $\Delta T$ was set to 0.1
and the swap period was set to 1,000 generations.
%
Runs went for 100 million generations,
sampling every 100,000 to produce a total of 1,000 samples.
%
We assumed that the taxon sampling fraction was 85\%
of the extant species diversity.


% State the results, drawing attention to important details
% in tables and figures
\section*{Results}

The results will go here.


% Point out the importance of the results and place them in the context
% of previous studies and in relation to the application of the work
% (expanding on the Synthesis and applications section of the Summary).
% Where appropriate, set out recommendations for management or policy
\section*{Discussion}

The discussion will go here.


\section*{Acknowledgements}

This research was supported in part through computational resources
and services provided by Advanced Research Computing
at the University of Michigan, Ann Arbor.


\bibliography{mc3}

\end{document}
