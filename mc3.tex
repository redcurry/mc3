% See author's guidelines for Methods in Ecology and Evolution:
% http://www.methodsinecologyandevolution.org/view/0/authorGuidelines.html

\documentclass[12pt]{article}

\usepackage{fullpage}

% No orphans or widows
\usepackage[all]{nowidow}

% Use mee bibliography style
\usepackage{natbib}
\bibliographystyle{mee}

\begin{document}

% Double-spaced
\baselineskip 24pt

\begin{flushleft}

{\Large\textbf{Metropolis coupled Markov chain Monte Carlo
    for detecting rate shifts on phylogenetic trees}}

% Short title must be <= 45 characters (current: 47, including spaces)
\textbf{Short title:} Metropolis coupled MCMC for rate shift detection

% Recommended: 6000-7000 words, including tables, figure captions, references
\textbf{Word count:} 0

Carlos J. R. Anderson$^{1}$ and
Daniel L. Rabosky$^{1,*}$

$^{1}$Department of Ecology and Evolutionary Biology,
    University of Michigan, Ann Arbor, MI 48109, USA

$^{*}$Corresponding author: Daniel L. Rabosky (drabosky@umich.edu)

\end{flushleft}


\pagebreak[4]


% Maximum of 350 words
\section*{Abstract}

The abstract will go here.

\begin{flushleft}
\textbf{Key-words:} BAMM, Bayesian, macroevolution, phylogeny, skink, speciation
\end{flushleft}


\pagebreak[4]


% State the reason for the work, the context and the hypotheses being tested
\section*{Introduction}

The introduction will go here.


% Include sufficient details for the work to be repeated
\section*{Materials and methods}

The methods will go here.


% State the results, drawing attention to important details
% in tables and figures
\section*{Results}

The results will go here.


% Point out the importance of the results and place them in the context
% of previous studies and in relation to the application of the work
% (expanding on the Synthesis and applications section of the Summary).
% Where appropriate, set out recommendations for management or policy
\section*{Discussion}

The discussion will go here.


\section*{Acknowledgements}

This research was supported in part through computational resources
and services provided by Advanced Research Computing
at the University of Michigan, Ann Arbor.


\end{document}
