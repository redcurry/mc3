% See author's guidelines for Methods in Ecology and Evolution:
% http://www.methodsinecologyandevolution.org/view/0/authorGuidelines.html

\documentclass[12pt]{article}

\usepackage{fullpage}

% No orphans or widows
\usepackage[all]{nowidow}

% Use mee bibliography style
\usepackage{natbib}
\bibliographystyle{mee}

% Definitions to make it easier to type
\newcommand{\MCMCMC}{(MC)$^{3}$}

\begin{document}

% Double-spaced
\baselineskip 24pt

\begin{flushleft}

{\Large\textbf{Metropolis coupled Markov chain Monte Carlo
    for detecting rate shifts on phylogenetic trees}}

% Short title must be <= 45 characters (current: 47, including spaces)
\textbf{Short title:} Metropolis coupled MCMC for rate shift detection

% Recommended: 6000-7000 words, including tables, figure captions, references
\textbf{Word count:} 0

Carlos J. R. Anderson$^{1}$ and
Daniel L. Rabosky$^{1,*}$

$^{1}$Department of Ecology and Evolutionary Biology,
    University of Michigan, Ann Arbor, MI 48109, USA

$^{*}$Corresponding author: Daniel L. Rabosky (drabosky@umich.edu)

\end{flushleft}


\pagebreak[4]


% Maximum of 350 words
\section*{Abstract}

\begin{flushleft}

\textbf{1.}
BAMM (Bayesian Analysis of Macroevolutionary Mixtures) is a computer program
that uses Markov chain Monte Carlo (MCMC) to model complex dynamics
of speciation, extinction, and trait evolution on phylogenetic trees.
%
Previous versions of BAMM used a single Markov chain
to explore the landscape of models and their parameter values.
%
A single chain, however, may get stuck in local optima,
resulting in less mixing and more time needed for convergence.

\textbf{2.}
Here we describe an extension to BAMM
that implements Metropolis coupled MCMC [\MCMCMC].
%
In \MCMCMC, additional chains are introduced
that are able to explore more of the model landscape.
%
Swaps between chains may occur periodically,
so if a chain is stuck in a local optimum,
it may immediately jump to another area of the landscape.
%
In addition, we describe multi-core \MCMCMC,
which is able to run each chain on a separate CPU.

\textbf{3.}
We find that in both simulated data sets and an empirical data set,
\MCMCMC\ mixes better and reduces the time to convergence
compared to a single-chain MCMC.
%
Greater than four chains results in diminishing returns.

\textbf{4.}
\MCMCMC\ greatly improves the reliability of results in BAMM,
and we recommend its use in all future studies of macroevolutionary dynamics.

\end{flushleft}

\begin{flushleft}
\textbf{Key-words:} BAMM, Bayesian, macroevolution, phylogeny, skink, speciation
\end{flushleft}


\pagebreak[4]


% State the reason for the work, the context and the hypotheses being tested
\section*{Introduction}

The introduction will go here, including discussion of BAMM.


% Include sufficient details for the work to be repeated
\section*{Materials and methods}

\subsection*{Metropolis coupled MCMC}

Describe implementation of MC$^{3}$.


\subsection*{Performance analysis}

Describe how performance was analyzed for simulated and emprical trees.


% State the results, drawing attention to important details
% in tables and figures
\section*{Results}

The results will go here.


% Point out the importance of the results and place them in the context
% of previous studies and in relation to the application of the work
% (expanding on the Synthesis and applications section of the Summary).
% Where appropriate, set out recommendations for management or policy
\section*{Discussion}

The discussion will go here.


\section*{Acknowledgements}

This research was supported in part through computational resources
and services provided by Advanced Research Computing
at the University of Michigan, Ann Arbor.


\bibliography{mc3}

\end{document}
